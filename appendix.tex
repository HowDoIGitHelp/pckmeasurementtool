\chapter*{Appendix}\label{appendix}
\addcontentsline{toc}{chapter}{Appendix}

\begin{landscape}
	\begin{longtable}[]{@{}p{100pt} p{40pt}p{160pt} p{160pt} p{40pt} @{}}
	\toprule
	\begin{minipage}[b]{0.12\columnwidth}\raggedright\strut
	Research Question\strut
	\end{minipage} & \begin{minipage}[b]{0.02\columnwidth}\raggedright\strut
	Participants\strut
	\end{minipage} & \begin{minipage}[b]{0.34\columnwidth}\raggedright\strut
	Sampling and Instruments\strut
	\end{minipage} & \begin{minipage}[b]{0.30\columnwidth}\raggedright\strut
	Analysis\strut
	\end{minipage} & \begin{minipage}[b]{0.07\columnwidth}\raggedright\strut
	Hypothesis\strut
	\end{minipage}\tabularnewline
	\midrule
	\endhead
	\begin{minipage}[t]{0.12\columnwidth}\raggedright\strut
	\textbf{\emph{Can the SCS1 test be repurposed into a PCK test?}}\strut
	\end{minipage} & \begin{minipage}[t]{0.02\columnwidth}\raggedright\strut
	x CS2 classes\strut
	\end{minipage} & \begin{minipage}[t]{0.34\columnwidth}\raggedright\strut
	Students will take the SCS1 pre test to compile common misconceptions.
	Based on these misconceptions a PCK test for teachers will be created to
	test if they can identify common misconceptions. Students will then take
	SCS1 post test at the end of the course.\strut
	\end{minipage} & \begin{minipage}[t]{0.30\columnwidth}\raggedright\strut
	Pearsons r correlation to measure the correlation between Teacher's PCK
	scores and student learning gains (SCS1 posttest scores - SCS1 pretest
	scores). Positive correlation means that SCS1 can be repurposed as PCK
	measurement tool.\strut
	\end{minipage} & \begin{minipage}[t]{0.07\columnwidth}\raggedright\strut
	Yes it can be used\strut
	\end{minipage}\tabularnewline
	\begin{minipage}[t]{0.12\columnwidth}\raggedright\strut
	\textbf{\emph{How does CS teachers? PCK relate to students?
	learning?}}\strut
	\end{minipage} & \begin{minipage}[t]{0.02\columnwidth}\raggedright\strut
	x CS2 classes\strut
	\end{minipage} & \begin{minipage}[t]{0.34\columnwidth}\raggedright\strut
	Students will take the SCS1 pre test to compile common misconceptions.
	Based on these misconceptions a PCK test for teachers will be created to
	test if they can identify common misconceptions. Students will then take
	SCS1 post test at the end of the course.\strut
	\end{minipage} & \begin{minipage}[t]{0.30\columnwidth}\raggedright\strut
	Pearsons r correlation to measure the correlation between Teacher's PCK
	scores and student learning gains (SCS1 posttest scores - SCS1 pretest
	scores)\strut
	\end{minipage} & \begin{minipage}[t]{0.07\columnwidth}\raggedright\strut
	High positive corelation between the two\strut
	\end{minipage}\tabularnewline
	\begin{minipage}[t]{0.12\columnwidth}\raggedright\strut
	\emph{How does CS teachers? content knowledge score relate to students?
	learning?}\strut
	\end{minipage} & \begin{minipage}[t]{0.02\columnwidth}\raggedright\strut
	x CS2 classes\strut
	\end{minipage} & \begin{minipage}[t]{0.34\columnwidth}\raggedright\strut
	Students will take the SCS1 pre test to compile common misconceptions.
	Teachers will take SCS1 test once. Students will then take SCS1 post
	test at the end of the course.\strut
	\end{minipage} & \begin{minipage}[t]{0.30\columnwidth}\raggedright\strut
	Pearsons r correlation to measure the correlation between Teacher's SCS1
	scores and student learning gains (SCS1 posttest scores - SCS1 pretest
	scores)\strut
	\end{minipage} & \begin{minipage}[t]{0.07\columnwidth}\raggedright\strut
	Low to zero correlation between the wo\strut
	\end{minipage}\tabularnewline
	\begin{minipage}[t]{0.19\columnwidth}\raggedright\strut
	\emph{How does CS teachers? ability to identify misconceptions relate to
	students? learning?}\strut
	\end{minipage} & \begin{minipage}[t]{0.19\columnwidth}\raggedright\strut
	x CS2 classes\strut
	\end{minipage} & \begin{minipage}[t]{0.19\columnwidth}\raggedright\strut
	Students will take the SCS1 pre test to compile common misconceptions.
	Based on these misconceptions a PCK test for teachers will be created to
	test if they can identify common misconceptions. Students will then take
	SCS1 post test at the end of the course.\strut
	\end{minipage} & \begin{minipage}[t]{0.19\columnwidth}\raggedright\strut
	Pearsons r correlation to measure the correlation between Teacher's PCK
	scores and student learning gains (SCS1 posttest scores - SCS1 pretest
	scores)\strut
	\end{minipage} & \begin{minipage}[t]{0.19\columnwidth}\raggedright\strut
	\strut
	\end{minipage}\tabularnewline
	\begin{minipage}[t]{0.19\columnwidth}\raggedright\strut
	\textbf{\emph{Which introductory programming concepts are hard to
	teach?}}\strut
	\end{minipage} & \begin{minipage}[t]{0.19\columnwidth}\raggedright\strut
	x CS2 teachers\strut
	\end{minipage} & \begin{minipage}[t]{0.19\columnwidth}\raggedright\strut
	Teachers will take PCK test tool created in this study.\strut
	\end{minipage} & \begin{minipage}[t]{0.19\columnwidth}\raggedright\strut
	\strut
	\end{minipage} & \begin{minipage}[t]{0.19\columnwidth}\raggedright\strut
	The same set as the topics students have difficulty on\strut
	\end{minipage}\tabularnewline
	\bottomrule
	\end{longtable}
\end{landscape}

\end{document}


